\documentclass{standalone}
\usepackage{tikz}
\usepackage[colorlinks=true]{hyperref}
\usetikzlibrary{
  arrows,
  calc,
  decorations.pathmorphing,
  decorations.pathreplacing,
  decorations.markings,
  fadings,
  positioning,
  shapes,
  arrows.meta
}
\tikzfading[name = lens fading,inner color = transparent!0,outer color = transparent!100]
\tikzfading[name = pbs fading,top color = transparent!0,bottom color = transparent!100]
\tikzset{
  mid arrow/.style={postaction={decorate,decoration={
        markings,
        mark=at position .5 with {\arrow[#1]{stealth}}
      }}},
  mid arrow2/.style={postaction={decorate,decoration={
        markings,
        mark=at position .5 with {\arrow[>=stealth]{><}}
      }}},
}

\newcommand\drawellipseshade[5][inner color=black,outer color=black]{
  % 1: shading option
  % 2: center (x, y)
  % 3: xsize
  % 4: ysize
  % 5: angle
  \begin{scope}
    \clip[rotate around={#5:#2}] #2 ellipse ({#3} and {#4});
    \begin{scope}[transform canvas={shift={#2}, rotate=#5}]
      \shade[shading=radial,path fading=lens fading, #1] (0, 0) ellipse ({#3} and {#4});
    \end{scope}
  \end{scope}
}
% \newcommand\drawlens[3]{
%   % 1: center (x, y)
%   % 2: size
%   % 3: angle
%   \drawellipseshade[inner color=blue,outer color=blue!40!cyan]{#1}{#2}{{0.14 * #2}}{#3}
% }
% \newcommand\drawwaveplate[3]{
%   % 1: center (x, y)
%   % 2: size
%   % 3: angle
%   \drawellipseshade[inner color=blue!50!black,outer color=blue!80!black]{#1}{#2}{{0.07 * #2}}{#3}
% }
% \newcommand\drawaom[4]{
%   % 1: center (x, y)
%   % 2: xsize
%   % 3: ysize
%   % 4: angle
%   \drawellipseshade[inner color=orange,outer color=orange]{#1}{#2}{#3}{#4}
%   \begin{scope}[rotate around={#4:#1}]
%     \fill[orange, even odd rule, opacity=0.8]
%     ($#1 + ({#2}, 0)$) arc (0:360:{#2} and {#3})
%     -- ($#1 + ({#2}, {#3})$) -- ($#1 + (-{#2}, {#3})$) -- ($#1 + (-{#2}, -{#3})$)
%     -- ($#1 + ({#2}, -{#3})$) --cycle;
%     \draw ($#1 + ({#2}, {#3})$) -- ($#1 + (-{#2}, {#3})$) -- ($#1 + (-{#2}, -{#3})$)
%     -- ($#1 + ({#2}, -{#3})$) --cycle;
%   \end{scope}
% }
% \newcommand\drawpbs[3]{
%   % 1: center (x, y)
%   % 2: size
%   % 3: angle
%   \begin{scope}
%     \begin{scope}
%       \clip[rotate around={#3:#1}] ($#1 - ({#2}, {#2})$) rectangle ($#1 + ({#2}, {#2})$);
%       \begin{scope}[transform canvas={shift={#1}, rotate=#3}]
%         \shade[bottom color=blue!60!cyan, top color=blue!50!cyan, path fading=pbs fading]
%         (-{#2}, -{#2}) rectangle ({#2}, {#2});
%         \draw[line width=1] (-{#2}, -{#2}) -- ({#2}, {#2});
%       \end{scope}
%     \end{scope}
%     % Make sure the frame is not clipped
%     \draw[rotate around={#3:#1}] ($#1 - ({#2}, {#2})$) rectangle ($#1 + ({#2}, {#2})$);
%   \end{scope}
% }
\newcommand\drawlens[3]{
  % 1: center (x, y)
  % 2: size
  % 3: angle
  \begin{scope}[shift={#1}]
    \node[rotate={#3}] at (0, 0) {\scalebox{#2}{\includegraphics[width=2cm]{lens.png}}};
  \end{scope}
}
\newcommand\drawwaveplate[3]{
  % 1: center (x, y)
  % 2: size
  % 3: angle
  \begin{scope}[shift={#1}]
    \node[rotate={#3}] at (0, 0) {\scalebox{#2}{\includegraphics[width=2cm]{wp.png}}};
  \end{scope}
}
\newcommand\drawaom[4]{
  % 1: center (x, y)
  % 2: xsize
  % 3: ysize
  % 4: angle
  \begin{scope}[shift={#1}]
    \node[rotate={#4}] at (0, 0)
    {\scalebox{#2}[#3]{\includegraphics[width=2cm,height=2cm]{AOM.png}}};
  \end{scope}
}
\newcommand\drawpbs[3]{
  % 1: center (x, y)
  % 2: size
  % 3: angle
  \begin{scope}[shift={#1}]
    \node[rotate={#3}] at (0, 0)
    {\scalebox{#2}{\includegraphics[width=1.4285714285714286cm]{PBS.png}}};
  \end{scope}
}
\newcommand\drawmirror[3]{
  % 1: center (x, y)
  % 2: size
  % 3: angle
  \draw[line width=3] ($#1 + ({#2 * cos(#3)}, {#2 * sin(#3)})$) --
  ($#1 - ({#2 * cos(#3)}, {#2 * sin(#3)})$);
}

\ifpdf
% Ensure reproducible output
\pdfinfoomitdate=1
\pdfsuppressptexinfo=-1
\pdftrailerid{}
\hypersetup{
  pdfcreator={},
  pdfproducer={}
}
\fi

\begin{document}

\begin{tikzpicture}
  \coordinate (Raman Input) at (2, 0);
  \coordinate (Raman Output) at (0, 2);
  \coordinate (DP PBS) at (0, 0);
  \coordinate (Wavemeter) at (0, -2);
  \coordinate (DP AOM) at (-2, 0);
  \coordinate (DP lens) at (-4.5, 0);
  \coordinate (DP M1) at (-6, 0);
  \coordinate (DP QWP) at (-6, 1.5);
  \coordinate (DP M2) at (-6 - 0.4, 2.5 + 0.05);
  \coordinate (SP QWP) at (-6, 3.5);
  \coordinate (SP IM1) at (-6, 5);
  \coordinate (SP lens) at (-7.5, 5);
  \coordinate (SP PBS) at (-9, 5);
  \coordinate (SP IM2) at (-11.5, 5);
  \coordinate (SP AOM) at (-11.5, 3.3);
  \coordinate (SP M1) at (-11.5, 0);
  \coordinate (SP M2) at (-9, 0);
  \coordinate (SP HWP) at (-9, 1.7);

  % Input
  \node[red,above] at (Raman Input) {\large\textbf{From Laser}};
  \draw[red,line width=1.6,mid arrow] (Raman Input) -- (DP PBS);
  % Wavemeter
  \draw[red,line width=1.6] (DP PBS) -- ($(DP PBS) + (0, -1)$);
  \draw[red,line width=1.6,mid arrow] ($(DP PBS) + (0, -1)$) -- (Wavemeter);
  \node[red,below] at (Wavemeter) {\large\textbf{To wavemeter}};
  % Between PBS and DP AOM
  \draw[red,line width=1.6,mid arrow2] (DP AOM) -- (DP PBS);
  \begin{scope}
    % Clip off one arrow
    \clip ($(DP AOM) + (0.77, 0)$) rectangle ($(DP AOM) + (0, 1)$);
    \draw[blue,line width=1.6,mid arrow2] (DP AOM) -- (DP PBS);
  \end{scope}
  \begin{scope}
    % Redraw the clipped line
    \clip (DP PBS) rectangle ($(DP AOM) + (0, 1)$);
    \draw[blue,line width=1.6] (DP AOM) -- (DP PBS);
  \end{scope}
  % Output
  \draw[red,line width=1.6] (DP PBS) -- ($(DP PBS) + (0, 1)$);
  \draw[red,line width=1.6,mid arrow] ($(DP PBS) + (0, 1)$) -- (Raman Output);
  \begin{scope}
    \clip (DP PBS) rectangle (-1, 3);
    \draw[blue,line width=1.6] (DP PBS) -- ($(DP PBS) + (0, 1)$);
    \draw[blue,line width=1.6,mid arrow] ($(DP PBS) + (0, 1)$) -- (Raman Output);
  \end{scope}
  \node[red] at ($(Raman Output) + (0, 0.3)$) {\large\textbf{To Experiment}};
  \begin{scope}
    \clip ($(Raman Output) + (-2, 0.3)$) rectangle ($(Raman Output) + (2, 1.5)$);
    \node[blue] at ($(Raman Output) + (0, 0.3)$) {\large\textbf{To Experiment}};
  \end{scope}
  \draw[red,line width=1.6,mid arrow2] (DP AOM) -- (DP M1);
  \draw[blue,line width=1.6,mid arrow2] (DP AOM) --
  node[above,pos=0.34,rotate=-5] {\small $\approx\!+345~\mathrm{MHz}$} ($(DP M1) + (-0.3, 0.3)$);
  \draw[blue,line width=1.6,mid arrow2] ($(DP M1) + (-0.3, 0.3)$) -- (DP M2);
  \draw[red,line width=1.6,mid arrow2] (DP M1) -- (SP IM1);
  \draw[red,line width=1.6,mid arrow2] (SP lens) -- (SP IM1);
  \draw[red,line width=1.6] (SP lens) -- (SP PBS);
  \draw[red,line width=1.6] (SP PBS) -- ($(SP PBS) + (-1, 0)$);
  \draw[red,line width=1.6,mid arrow] ($(SP PBS) + (-1, 0)$) -- (SP IM2);
  \draw[red,line width=1.6] (SP IM2) -- (SP AOM);
  \draw[red,line width=1.6,mid arrow] (SP AOM) --
  node[above,rotate=86,pos=0.45] {\small $-80~\mathrm{MHz}$} ($(SP M1) + (-0.2, 0.2)$);
  \draw[red,line width=1.6,mid arrow] ($(SP M1) + (-0.2, 0.2)$) -- (SP M2);
  \draw[red,line width=1.6,mid arrow] (SP M2) -- (SP PBS);
  % Input PBS
  \drawpbs{(DP PBS)}{0.7}{0}
  \node[blue!40!cyan,below right] at ($(DP PBS) + (0.7, -0.7)$) {\large PBS 1};
  % DP AOM
  \drawaom{(DP AOM)}{1}{0.5}{90}
  \node[rotate=-90] at (DP AOM) {\large DP AOM};
  % DP Lens
  \drawlens{(DP lens)}{1}{90}
  \node[blue!80!cyan,below] at ($(DP lens) + (0, -1)$-4.5, -1) {\large L1};
  \drawmirror{(DP M1)}{0.9}{-45}
  % DP QWP
  \drawwaveplate{(DP QWP)}{1}{0}
  \node[blue!80!black,right] at ($(DP QWP) + (0.9, 0)$) {\large $\lambda/4$};
  \drawmirror{(DP M2)}{0.3}{3.8}
  % Post DP QWP
  \drawwaveplate{(SP QWP)}{1}{0}
  \node[blue!80!black,right] at ($(SP QWP) + (0.9, 0)$) {\large $\lambda/4$};
  \drawmirror{(SP IM1)}{0.9}{-45}
  % Post DP Lens
  \drawlens{(SP lens)}{1}{90}
  \node[blue!80!cyan,above] at ($(SP lens) + (0, 1)$) {\large L2};
  % SP PBS
  \drawpbs{(SP PBS)}{0.7}{0}
  \node[blue!40!cyan,above] at ($(SP PBS) + (0, 0.7)$) {\large PBS 2};
  \drawmirror{(SP IM2)}{0.9}{45}
  % SP AOM
  \drawaom{(SP AOM)}{1}{0.5}{0}
  \node at (SP AOM) {\large SP AOM};
  \drawmirror{(SP M1)}{0.9}{-45}
  \drawmirror{(SP M2)}{0.9}{45}
  % SP HWP
  \drawwaveplate{(SP HWP)}{1}{0}
  \node[blue!80!black,left] at ($(SP HWP) + (-0.9, 0)$) {\large $\lambda/2$};
\end{tikzpicture}

\end{document}
