\documentclass[aps,secnumarabic,amsmath,amssymb,superscriptaddress]{revtex4}
\usepackage{amsmath}
\usepackage{amssymb}
\usepackage{amsfonts}
\usepackage{color}
\usepackage{graphics}
\usepackage[pdftex]{graphicx}
\usepackage[utf8x]{inputenc}
\usepackage[colorlinks=true]{hyperref}
\usepackage{listings}

\newcommand{\ud}{\mathrm{d}}
\newcommand{\ue}{\mathrm{e}}
\newcommand{\ui}{\mathrm{i}}
\newcommand{\res}{\mathrm{Res}}
\newcommand{\Tr}{\mathrm{Tr}}
\newcommand{\dsum}{\displaystyle\sum}
\newcommand{\dprod}{\displaystyle\prod}
\newcommand{\dlim}{\dispqlaystyle\lim}
\newcommand{\dint}{\displaystyle\int}
\newcommand{\fsno}[1]{{\!\not\!{#1}}}
\newcommand{\texp}[2]{\ensuremath{{#1}\times10^{#2}}}
\newcommand{\dexp}[2]{\ensuremath{{#1}\cdot10^{#2}}}
\newcommand{\eval}[2]{{\left.{#1}\right|_{#2}}}
\newcommand{\paren}[1]{{\left({#1}\right)}}
\newcommand{\lparen}[1]{{\left({#1}\right.}}
\newcommand{\rparen}[1]{{\left.{#1}\right)}}
\newcommand{\abs}[1]{{\left|{#1}\right|}}
\newcommand{\sqr}[1]{{\left[{#1}\right]}}
\newcommand{\crly}[1]{{\left\{{#1}\right\}}}
\newcommand{\angl}[1]{{\left\langle{#1}\right\rangle}}
\newcommand{\tpdiff}[4][{}]{{\paren{\frac{\partial^{#1} {#2}}{\partial {#3}{}^{#1}}}_{#4}}}
\newcommand{\tpsdiff}[4][{}]{{\paren{\frac{\partial^{#1}}{\partial {#3}{}^{#1}}{#2}}_{#4}}}
\newcommand{\pdiff}[3][{}]{{\frac{\partial^{#1} {#2}}{\partial {#3}{}^{#1}}}}
\newcommand{\diff}[3][{}]{{\frac{\ud^{#1} {#2}}{\ud {#3}{}^{#1}}}}
\newcommand{\psdiff}[3][{}]{{\frac{\partial^{#1}}{\partial {#3}{}^{#1}} {#2}}}
\newcommand{\sdiff}[3][{}]{{\frac{\ud^{#1}}{\ud {#3}{}^{#1}} {#2}}}
\newcommand{\tpddiff}[4][{}]{{\left(\dfrac{\partial^{#1} {#2}}{\partial {#3}{}^{#1}}\right)_{#4}}}
\newcommand{\tpsddiff}[4][{}]{{\paren{\dfrac{\partial^{#1}}{\partial {#3}{}^{#1}}{#2}}_{#4}}}
\newcommand{\pddiff}[3][{}]{{\dfrac{\partial^{#1} {#2}}{\partial {#3}{}^{#1}}}}
\newcommand{\ddiff}[3][{}]{{\dfrac{\ud^{#1} {#2}}{\ud {#3}{}^{#1}}}}
\newcommand{\psddiff}[3][{}]{{\frac{\partial^{#1}}{\partial{}^{#1} {#3}} {#2}}}
\newcommand{\sddiff}[3][{}]{{\frac{\ud^{#1}}{\ud {#3}{}^{#1}} {#2}}}
\newcommand{\eff}{ef\! f}

\newcommand{\todo}[1]{}

\newcommand{\harvardphysics}{\affiliation{Department of Physics, Harvard University, Cambridge, Massachusetts 02138, USA}}
\newcommand{\harvardccb}{\affiliation{Department of Chemistry and Chemical Biology, Harvard University, Cambridge, Massachusetts 02138, USA}}
\newcommand{\cua}{\affiliation{Harvard-MIT Center for Ultracold Atoms, Cambridge, Massachusetts 02138, USA}}
\newcommand{\gradstudent}{
  \harvardphysics
  \harvardccb
  \cua
}

% Add S1 to bibliography
\bibliographystyle{apsrev4-2}
\renewcommand*{\citenumfont}[1]{S#1}
\renewcommand*{\bibnumfmt}[1]{[S#1]}

% Add S to labels
\setcounter{table}{0}
\renewcommand{\thetable}{S\arabic{table}}%
\setcounter{figure}{0}
\renewcommand{\thefigure}{S\arabic{figure}}%
\renewcommand{\thepage}{S\arabic{page}}
\renewcommand{\thesection}{S\arabic{section}}
\renewcommand{\theequation}{S.\arabic{equation}}

\ifpdf
% Ensure reproducible output
\pdfinfoomitdate=1
\pdfsuppressptexinfo=-1
\pdftrailerid{}
\hypersetup{
  pdfcreator={},
  pdfproducer={}
}
\fi

\begin{document}
\title{Coherent optical association of a single molecule -- Supplemental material}
\author{Yichao~Yu}
\email{yichaoyu@g.harvard.edu}
\gradstudent
\author{Kenneth~Wang}
\gradstudent
\author{Jonathan~D.~Hood}
\affiliation{Department of Chemistry, Purdue University, West Lafayette, Indianna, 47906}
\author{Lewis~R.~B.~Picard}
\gradstudent
\author{Jessie~T.~Zhang}
\gradstudent
\author{William~B.~Cairncross}
\harvardccb
\harvardphysics
\cua
\author{Jeremy~M.~Hutson}
\affiliation{Joint Quantum Centre Durham-Newcastle, Department of Chemistry, Durham University, Durham, DH1 3LE, United Kingdom}
\author{Till Rosenband}
\harvardphysics
\author{Kang-Kuen~Ni}
\email{ni@chemistry.harvard.edu}
\harvardccb
\harvardphysics
\cua

\date{\today}

\maketitle

\section{3 Level Raman Transfer with Cross Coupling}

In our system, the separation in energy between the initial and target state is much smaller than the single photon detuning $\Delta$ from the intermediate state. Thus, there is significant cross coupling for the scattering rate and light shift, where each state is coupled to the excited state by the power in both beams. Under these conditions, the scattering rate and light shift is proportional to the total power $P_{\text{tot}} = P_m + P_a$, where $P_{m} (P_{a})$ is the power in the beam that addresses the molecular (atomic) state. The Raman Rabi frequency is proportional to $\sqrt{P_mP_a} $. With a fixed total power and thus fixed scattering rate, the Raman Rabi frequency is maximized by maximizing $\sqrt{P_m(P_{\text{tot}} - P_m)}$, which is achieved when $ P_m = P_a = P_{\text{tot}}/2$.

\section{Raman Transfer with Many Excited States}

\section{Calculating Matrix Elements with Coupled Channel Ground State}

Test Reference\cite{Liu2019}

\todo{
  Power/intensity calibration
}

\bibliography{master_ref}
\end{document}
