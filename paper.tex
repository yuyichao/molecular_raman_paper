\documentclass[aps,prl,twocolumn,10pt,superscriptaddress]{revtex4-1}
% \documentclass[aps,twocolumn,secnumarabic,balancelastpage,amsmath,amssymb,nofootinbib]{revtex4-1}
\usepackage{amsmath}
\usepackage{amssymb}
\usepackage{amsfonts}
\usepackage{color}
\usepackage{graphics}
\usepackage[pdftex]{graphicx}
\usepackage[utf8x]{inputenc}
\usepackage[colorlinks=true]{hyperref}
\usepackage{footmisc}
\usepackage{braket}

\newcommand{\ud}{\mathrm{d}}
\newcommand{\ue}{\mathrm{e}}
\newcommand{\ui}{\mathrm{i}}
\newcommand{\res}{\mathrm{Res}}
\newcommand{\Tr}{\mathrm{Tr}}
\newcommand{\dsum}{\displaystyle\sum}
\newcommand{\dprod}{\displaystyle\prod}
\newcommand{\dlim}{\displaystyle\lim}
\newcommand{\dint}{\displaystyle\int}
\newcommand{\fsno}[1]{{\!\not\!{#1}}}
\newcommand{\texp}[2]{\ensuremath{{#1}\times10^{#2}}}
\newcommand{\dexp}[2]{\ensuremath{{#1}\cdot10^{#2}}}
\newcommand{\eval}[2]{{\left.{#1}\right|_{#2}}}
\newcommand{\paren}[1]{{\left({#1}\right)}}
\newcommand{\lparen}[1]{{\left({#1}\right.}}
\newcommand{\rparen}[1]{{\left.{#1}\right)}}
\newcommand{\abs}[1]{{\left|{#1}\right|}}
\newcommand{\sqr}[1]{{\left[{#1}\right]}}
\newcommand{\crly}[1]{{\left\{{#1}\right\}}}
\newcommand{\angl}[1]{{\left\langle{#1}\right\rangle}}
\newcommand{\tpdiff}[4][{}]{{\paren{\frac{\partial^{#1} {#2}}{\partial {#3}{}^{#1}}}_{#4}}}
\newcommand{\tpsdiff}[4][{}]{{\paren{\frac{\partial^{#1}}{\partial {#3}{}^{#1}}{#2}}_{#4}}}
\newcommand{\pdiff}[3][{}]{{\frac{\partial^{#1} {#2}}{\partial {#3}{}^{#1}}}}
\newcommand{\diff}[3][{}]{{\frac{\ud^{#1} {#2}}{\ud {#3}{}^{#1}}}}
\newcommand{\psdiff}[3][{}]{{\frac{\partial^{#1}}{\partial {#3}{}^{#1}} {#2}}}
\newcommand{\sdiff}[3][{}]{{\frac{\ud^{#1}}{\ud {#3}{}^{#1}} {#2}}}
\newcommand{\tpddiff}[4][{}]{{\left(\dfrac{\partial^{#1} {#2}}{\partial {#3}{}^{#1}}\right)_{#4}}}
\newcommand{\tpsddiff}[4][{}]{{\paren{\dfrac{\partial^{#1}}{\partial {#3}{}^{#1}}{#2}}_{#4}}}
\newcommand{\pddiff}[3][{}]{{\dfrac{\partial^{#1} {#2}}{\partial {#3}{}^{#1}}}}
\newcommand{\ddiff}[3][{}]{{\dfrac{\ud^{#1} {#2}}{\ud {#3}{}^{#1}}}}
\newcommand{\psddiff}[3][{}]{{\frac{\partial^{#1}}{\partial{}^{#1} {#3}} {#2}}}
\newcommand{\sddiff}[3][{}]{{\frac{\ud^{#1}}{\ud {#3}{}^{#1}} {#2}}}
\newcommand{\eff}{ef\! f}
\newcommand{\Na}{\mathrm{Na}}
\newcommand{\Cs}{\mathrm{Cs}}
\newcommand{\fxnote}[1]{{\textbf{[#1]}}}

\newcommand{\todo}[1]{}

\ifpdf
% Ensure reproducible output
\pdfinfoomitdate=1
\pdfsuppressptexinfo=-1
\pdftrailerid{}
\hypersetup{
  pdfcreator={},
  pdfproducer={}
}
\fi

\newcommand{\harvardphysics}{\affiliation{Department of Physics, Harvard University, Cambridge, Massachusetts 02138, USA}}
\newcommand{\harvardccb}{\affiliation{Department of Chemistry and Chemical Biology, Harvard University, Cambridge, Massachusetts 02138, USA}}
\newcommand{\cua}{\affiliation{Harvard-MIT Center for Ultracold Atoms, Cambridge, Massachusetts 02138, USA}}
\newcommand{\gradstudent}{
  \harvardphysics
  \harvardccb
  \cua
}

\begin{document}
\title{Coherent optical association of a single molecule}
\author{Yichao~Yu}
\email{yichaoyu@g.harvard.edu}
\gradstudent
\author{Kenneth~Wang}
\gradstudent
\author{Jonathan~D.~Hood}
\affiliation{Department of Chemistry, Purdue University, West Lafayette, Indianna, 47906}
\author{Lewis~R.~B.~Picard}
\gradstudent
\author{Jessie~T.~Zhang}
\gradstudent
\author{William~B.~Cairncross}
\harvardccb
\harvardphysics
\cua
\author{Jeremy~M.~Hutson}
\affiliation{Joint Quantum Centre Durham-Newcastle, Department of Chemistry, Durham University, Durham, DH1 3LE, United Kingdom}
\author{Till Rosenband}
\harvardphysics
\author{Kang-Kuen~Ni}
\email{ni@chemistry.harvard.edu}
\harvardccb
\harvardphysics
\cua

\date{\today}

\begin{abstract}
  Fully controlled single molecules provide a promising platform for a variety of quantum science applications. As a step towards this goal, we report on coherent association of a single weakly-bound NaCs molecule in an optical tweezer
  through an optical Raman transition without the use of a Feshbach resonance.
  The Raman scheme uses a deeply bound electronic excited intermediate state
  to achieve a large transition dipole moment while reducing the photon scattering.
  Starting from two atoms in their relative motional ground state,
  we achieve an optical transfer efficiency of $69~\mathrm{\%}$.
  The molecule has a binding energy of $770.20052(2)~\mathrm{MHz}$ at $8.8~\mathrm{G}$ 
  with more than $60~\mathrm{\%}$ of the molecule created in the motional ground state.
  This technique does not rely on narrow excited state lines or Feshbach resonances
  and will allow a wider range of molecular species to be assembled atom-by-atom.
\end{abstract}

\maketitle

% Introduction


% 1. we want to have a diverse species to tailor to different applications including precision measurements and quantum engineering. 2. for the technique of associating atoms to form molecules, but so far all has been done with Feshbach association (before STIRAP), or in the case of Sr2, a narrow excited state. Previous work on near-threshold Raman transfer has been incoherent.
% (suggested referees: Florian Shreck and Immanual Bloch, Tanya, Deep Gupta, DeMille)

% Trapped neutral molecules, assembled in an array of optical tweezers, are a promising platform to study quantum information and quantum simulation. (more detail to add here about applications?).

Diverse species of fully quantum controlled  ultracold molecules are desired for a  wide variety of applications including precision measurements~\cite{Kondov2019,Nick_and_Ivan2017, PhysRevA.101.042504, Andreev2018, PhysRevLett.119.153001, hudson2011}, quantum simulations~\cite{Micheli2006, Yao2018, Wall2015, wall2015realizing}, quantum information processing~\cite{DeMille2002, Ni2018, Hudson2018, Lin2019}, and studies of ultracold chemistry~\cite{Bala2016,Hu1111,Segev2019,deJongh626}.
While many innovative approaches demonstrated in the last few years have directly cooled different species of diatomic or polyatomic molecules below 1~mK~\cite{Norrgard2016,Anderegg2018, Mitra1366, PhysRevX.10.021049, PhysRevLett.121.013202, Truppe2017}, the coldest and the highest phase-space-density gas to date in an ensemble~\cite{Demarco2018} or as individuals~\cite{Zhang2020,He331}  have been achieved through the association of ultracold atoms.
% (any place to cite  molecular ion work?)
% Citations are not complete. Add them as you see fit.

Molecular association of ultracold atoms takes advantage of the cooling and trapping techniques that have been developed for atoms. Associating atoms into deeply-bound molecules is challenging because of the small wavefunction overlap between the free-atom and molecular state and the release of a large amount of binding energy. A two step approach has been demonstrated to associate atom pairs into weakly bound molecules first, and then transfer the molecules from this single internal state to a desired rovibrational and electronic state releasing the binding energy via stimulated emission~\cite{Danzl2008, Ni2008,Lang2008, Takekoshi2014, Molony2014, Park2015, Guo2016, Kondov2019, Voges2020}.
So far, most molecular association has been achieved for bialkali molecules by magnetoassociation using a magnetic Feshbach scattering resonance. The only exceptions are Sr$_2$, where narrow linewidth ($\sim 20~\mathrm{kHz}$) excited states are available and optical association can be driven coherently,~\cite{Reinaudi2012,Stellmer2012} and $^{87}$Rb$^{85}$Rb, where weakly-bound molecules with a couple MHz binding energy exists~\cite{He331}. With these requirements, molecules involving non-magnetic atoms or atoms without narrow intercombination lines remain difficult to associate.
% These requirements limit the generality of previous association techniques.
% The requirement of a Feshbach resonance or a MHz-level binding energy bound state to enhance atom-to-molecule wavefunction overlap or the existence of narrow excited state lines limits the generality of previous association techniques. % to more diverse species.



Here, we demonstrate coherent association of an atom pair to a weakly bound molecule using a two-photon optical Raman transfer via an electronic excited state, schematically shown in Fig.~\ref{f-theory}a. The scheme does not rely on a Feshbach resonance, few MHz level bound states, or a narrow excited state. The resulting single molecule is in a well-defined internal quantum state and predominately in its motional ground state. A vibrational state of the electronic excited state $\mathrm{c^3\Sigma^+}(\Omega = 1)$ is used as the intermediate state in the Raman scheme, and is chosen to minimize the theoretical photon scattering during a Raman Rabi oscillation. To further reduce the photon scattering and sensitivity to laser intensity noise, we choose the initial and final state to balance the two Rabi frequencies as much as possible. This system-independent approach can be used for creating new molecules atom-by-atom with full quantum state control.


\begin{figure*}
  \includegraphics[width=\textwidth]{fig1.pdf}
  \caption{Optical creation of single molecule from single atoms in tweezer.
    (a) Schematic of the optical transition from an atom pair to a weakly-bound molecule.
    The initial state is the relative motional ground state between the two atoms
    and the final state is the first molecular bound state.
    The transition is driven by a pair of laser frequencies matching the binding energy
    of the molecule.
    The lasers are detuned from an excited molecular state in the $\mathrm{c^3\Sigma}$ potential
    by $\Delta$ in order to suppress the scattering during the transfer.
    (b) Comparison between using a weakly-bound and a deeply bound excited state
    as intermediate state for the Raman transition.
    The deeply bound excited state (upper half $v'=0$)
    has a smaller Raman Rabi frequency ($\Omega_{R}$)
    compared to the weakly-bound excited state (lower half $v'=63$) at a given detuning.
    However, the scattering rate ($\Gamma_{s}$) is also much lower,
    which results in a larger Raman Rabi frequency to scattering rate ratio.
    (c) Enhancement of the short range wavefunction.
    The large scattering length for the $\Na(2,2),\Cs(3,3)$ state creates an interaction shift
    comparable to the axial trapping frequency.
    This causes a significant change in the relative wavefunction especially at short
    intranuclear distance ($R$).
    Compared to other spin states with weaker interaction,
    the wavefunction at short distance ($R<100\text{\AA}$, left of the dotted line)
    is significantly enhanced.
    \label{f-theory}
  }
\end{figure*}



The essence of an optical Raman transfer can be illustrated using a three-level system (Fig.~\ref{f-theory}a), where the initial atomic state and the target weakly bound molecular state are coupled to an intermediate state by two lasers with Rabi frequencies, $\Omega_a$ and $\Omega_m$, with one-photon detuning $ \Delta $, and two-photon detuning, $ \delta$.  %(Ultimately, we calculate with contribution of all vibrational states of the excited electronic potential, $\mathrm{c^3\Sigma^+}$)
The transfer Raman Rabi Rate, $\Omega_a\Omega_m / 2\Delta$, is accompanied by a photon scattering rate $\Gamma_e (\Omega_a^2 + \Omega_m^2)/4\Delta^2$
% $\Gamma_e \Omega^2 / 4\Delta^2 $, where $ \Gamma_e $ is the excited state linewidth
~\cite{Wineland2003}.
Unlike Raman transitions in atoms, the two Rabi frequencies are greatly imbalanced %. Since $ \Omega_m $ is generally much larger than $ \Omega_a $
due to the small wavefunction overlap between the atomic state and the intermediate state, %the atomic state and the excited molecular state,
and therefore the scattering is predominantly from the target molecular state. Furthermore, the energy difference between the atomic state and target molecular state is small ($ < 1~\mathrm{GHz} $) compared to the single photon detuning, $ \Delta $, so the target molecular state can scatter off both beams roughly equally. Thus, the scattering rate is given by $ \Gamma_e \Omega_m^2 / 2\Delta^2$, where $ \Gamma_e $ is the excited state linewidth \footnote{We choose the two beams to have equal power, which gives the highest Raman Rabi rate at a fixed total power. Thus, this results in a simple factor of 2 coming from scattering off 2 beams.}.
The ratio between the Raman Rabi frequency and the scattering rate is therefore $ \Omega_a/\Omega_m \times \Delta/\Gamma_e $. %, depends on the ratio of the two matrix elements and how far detuned the laser is from the transition in units of the linewidth.
To ensure a coherent process, a detuning as large as possible, while maintaining a realistic Raman Rabi frequency, is preferred. %However, the detuning cannot be too large, since that will reduce the Raman Rabi frequency.

Early experiments used weakly bound excited states as the intermediate state
in the Raman transition to ensure a large Raman Rabi frequency~\cite{Wynar2000,Rom2004}.
However, for a complete picture, the many excited vibrational states in a molecular potential
as well as the excited atomic continuum need to be considered.
The total scattering rate and Raman Rabi rate become a sum of the scattering rates
and Raman Rabi rates over all possible intermediate states.
With these considerations, using a weakly bound excited state as the intermediate state
suffers from strong scattering of the nearby excited atomic continuum, resulting in large incoherence and loss to other molecular states.
This scattering is proportional to $1/\delta_{\mathrm{thresh}}^2$,
where $\delta_{\mathrm{thresh}}$ is the detuning from the dissociation threshold,
and thus can be made smaller by using deeply bound vibrational states as the intermediate level.


To find the optimal intermediate state, we perform a calculation of the Raman Rabi frequency $\Omega_R$
and scattering rate $\Gamma_s$ at different detunings from the atomic threshold
taking into account of all states of
the $\mathrm{c^3\Sigma^+}(\Omega = 1)$ excited molecular state potential~\cite{Grochola2011}
and the continuum~\cite{Liu2017}.
The excited atomic continuum is particularly important for the target molecular state.
The sum of the squares of the wavefunction overlap
between the target weakly-bound molecular state
and all the excited molecular bound states is only about $0.02$,
suggesting that there is significant matrix element
between the target molecular state and the excited atomic continuum (See Supplementary Material\todo{proper cite}).
This calculation shows that the ratio of the Raman Rabi rate to scattering rate
can be made larger for more deeply bound states compared to weakly bound states
at a cost of a smaller Raman Rabi frequency
(Fig.~\ref{f-theory}b, full result in Supplementary Material \todo{proper cite})
As a result, we choose the $v'=0$ of $\mathrm{c^3\Sigma^+}(\Omega = 1)$ as an intermediate state
to drive the Raman transition.

In addition to the intermediate state,
choosing an initial and a final state for a large $ \Omega_a/\Omega_m $ ratio
maximizes the Raman Rabi frequency at a given detuning.
Furthermore, a larger ratio also relaxes the intensity stability requirement,
because this is also the ratio between the Raman Rabi coupling
and the AC Stark shift of the molecular state, $\Omega_m^2 / 2\Delta$
\footnote{There is an additional factor of 2, with both beams at equal power,
  to account for the Stark shift caused by both beams.}.
Due to the small size of the intermediate state wavefunction as compared to the trapped atoms,
$\Omega_a$ is approximately proportional to
the value of the relative atomic wavefunction at short distance
within the molecular potential.
To increase the amplitude within the molecular potential,
one can increase the external confinement of atom pairs.
Using a harmonic oscillator approximation,
the short range amplitude is proportional to $ \omega_{\text{trap}}^{3/4} $ or $P^{3/8}$,
where $ \omega_{\text{trap}} $ is the trap frequency and $P$ is the power in the trap\cite{Mies2000}. However, additional power may not always be available, and will also lead to additional undesired scattering.
Alternatively, one can choose an atomic pair state with a large scattering length
(positive or negative).
For these states, the phase shift in the relative wavefunction
between the atoms can significantly increase the short range wavefunction (Fig.~\ref{f-theory}c).
% The increase in the coupling is proportional to (quote/cite Olive's equation?).
For our system of Na and Cs atoms,
we choose a spin state combination $\ket{\uparrow_{\Na} \downarrow_{\Cs}}\equiv \ket{F=2,m_F=2}_{\Na}\ket{F=3,m_F=3}_{\Cs}$ that has a large and negative scattering length of
$ a(\uparrow_{\Na} \downarrow_{\Cs}) = -693.8a_0$~\cite{Hood2019}.
All other stable spin combinations give smaller scattering lengths ($<~50~a_0$).
% We denote the possible hyperfine states of the atoms as $ \ket{\uparrow_{\Cs}} = \ket{F=4,m_F=4}_{\Cs}, \ket{\downarrow_{\Cs}} = \ket{F=3,m_F=3}_{\Cs}, \ket{\uparrow_{\Na}} = \ket{F=2,m_F=2}_{\Na}, $ and $ \ket{\downarrow_{\Na}} = \ket{F=1,m_F=1}_{\Na}$. Among the stable spin combinations, $\ket{\uparrow_{\Na}\uparrow_{\Cs}}$ and $\ket{\downarrow_{\Na}\downarrow_{\Cs}} $ both have small scattering lengths of $ 30.4a_0 $, and $ 13.7a_0 $ respectively, but the $\ket{\uparrow_{\Na} \downarrow_{\Cs}} $ combination has a large and negative scattering length of $ a(\uparrow_{\Na} \downarrow_{\Cs}) = -693.8a_0 $ (interaction shift $\approx$ binding?)~\cite{Hood2019}.
For the target molecular state, the spin state is ideally similar to the initial spin state in order to minimize sensitivity to the magnetic field.
% We find a target spin state predominantly in the initial spin state
% that also has the advantage of having a reduced sensitivity
% to laser intensity noise because of a larger %$\Omega_a/\Omega_b$ as compared to other states.
% This reduces the intensity stability requirement to $5~\mathrm{\%}$
% instead of $0.3~\mathrm{\%}$ which is typical for other states.
% In addition to the increased atomic coupling, $ \Omega_a $,
Coupled channel calculations show that this target molecular state with similar spin composition also has reduced Rabi frequency, $ \Omega_m $,
with the intermediate state when compared to bound states of the other spin compositions. This further increases the $\Omega_a/\Omega_m$ ratio when using the $\ket{\uparrow_{\Na} \downarrow_{\Cs}}$ hyperfine combination.
% $\ket{\uparrow_{\Na}\uparrow_{\Cs}}$ and $\ket{\downarrow_{\Na}\downarrow_{\Cs}} $ bound states.
This combination
results in a $ \Omega_a/\Omega_m$ ratio of about 0.05 instead of a ratio of about 0.003
with the other combinations, thus relaxing the intensity stability requirement to the few percent level. % Thus, instead of a required intensity stability of $0.3 ~\mathrm{\%} $ for the 4422 or 3311 combination, the 3322 combination only requires the intensity to be stabilized to $ 5~\mathrm{\%} $.
% Thus, we choose the $\ket{\uparrow_{\Na} \downarrow_{\Cs}}$ spin combination as our initial state and drive to the first bound state for the $\ket{\uparrow_{\Na} \downarrow_{\Cs}}$ spin combination.
% \textcolor{blue}{(clean up the paragraph some more).} %(We might want to mention that those ratios are for 80 kHz spherical trap if we want to give more details about the coupled channel calculation anywhere...)

% In additional to the final and the excited state, it is also important to select the an initial atomic state in order to improve the coupling.
\begin{figure}[ht!]
  \includegraphics[width=0.48\textwidth]{fig2.pdf}
  \caption{
    (a) Raman detuning scans at different times showing the resonance frequency.
    (b) Raman pulse time scan on resonance.
    A decaying Rabi oscillation can be observed proving the coherence of
    the Raman transfer process.
    This is fitted together with (a) to a model of the Raman transition including loss of the atom and molecule state and is used to determine
    both the Raman Rabi frequency and the loss rates.
    Inset: Geometry and polarization of trap and Raman beam relative to the bias magnetic field.
    The tweezer and Raman beam is focused through an objective to a waist of $0.9~\mathrm{\mu m}$
    which defines the location of the atoms and molecule.
    We use a bias B field of $B_0=8.8~\mathrm{G}$ along the tweezer polarization
    to define the quantization axis.
    As a result, the atoms experiences predominately $\pi$ polarization from the tweezer.
    \label{f-raman}}
\end{figure}




%% Preparation
Experimentally, we first prepare two atoms in a well-defined external and internal quantum state using techniques developed previously~\cite{Liu2018, Liu2019, Wang2019}. In brief, the experimental cycle begins by stochastically loading a single ${}^{23}\Na$ atom and a single ${}^{133}\Cs$ atom into separate optical tweezers. The atoms are initially imaged to distinguish between loading of two atoms, one atom (Na or Cs), or no atom to be able to post select the experimental result on the initial loading condition. Raman sideband cooling is then applied to prepare both atoms simultaneously into the 3-dimensional motional ground state of their optical tweezers. After cooling, the Na and Cs atoms are in a spin state of $\ket{\uparrow_{\Na}\uparrow_{\Cs}}\equiv \ket{F=2,m_F=2}_{\Na}\ket{F=4,m_F=4}_{\Cs} $ %and are merged into the same tweezer~\cite{Liu2019}. %The spin states for the Na and Cs atoms after RSC and during the merge process are $\ket{\uparrow_{\Na}\uparrow_{\Cs}} $.
which has a small scattering length. The small two atom interaction allows the merging of the two atoms to be done with minimum pertubation so that they remain in the motional ground state.

% After preparing the Na and Cs atoms in the same tweezer in a single quantum state,
Next, we drive the atoms into the large scattering length $\ket{\uparrow_{\Na} \downarrow_{\Cs}}$ hyperfine combination to use as the initial atomic state of Raman transfer by performing a Cs spin flip while taking into account the $-30.7~\mathrm{kHz}$ interaction shift~\cite{Hood2019}.  %using a Cs Raman transition to drive the Cs atom into the $\ket{\downarrow_{\Cs}}$ state. The new spin state combination has a larger scattering length of $-693.8 a_0$ which generates a interaction shift of $-30.7 kHz$ in the tweezer.
This spin flip selectively transfers atoms in the relative motional ground state, removing any initial hot atom background from imperfect cooling \footnote{This interaction shift is larger than the differential axial trapping frequency between Na and Cs atoms, which decouples the relative and center of mass motional state and improves the robustness of our preparation of the relative motional ground state.}. For the experiment reported here, $31~\mathrm{\%}$ of our initial two atom population is successfully transferred.
% of two atoms loaded in separate optical tweezers.\textcolor{red}{I also find this part of the sentence confusing.  can you say this more concisely?}\todo{}
% The stronger interaction in this spin state also enhances the atomic wavefunction at short range and increases its overlap with the intermediate molecular state used for our Raman transfer..

%% Transfer scheme
% After the atoms are prepared in the $\ket{\uparrow_{\Na}\uparrow_{\Cs}} $ hyperfine combination, we then perform the Raman transfer.
To perform the Raman transfer of an atom pair to the target weakly bound molecular state, we use the tweezer beam itself as the Raman beams by turning on two frequencies in the tweezer during the Raman pulse (Fig.~\ref{f-raman}b inset). The dual use of the tweezer beam not only eliminates additional scattering sources or undesired laser frequencies, but also allows a tight focus to maximize the Raman Rabi frequency and minimize the transfer time. Furthermore, we use a Bragg grating with a linewidth (FWHM) of $50 ~\mathrm{GHz}$ to filter the laser spectrum generated by a fiber amplifier that is seeded with a $1037~\mathrm{nm}$ external cavity diode laser. We observed a reduction of the scattering rate by a factor of $2$ due to suppression of the broadband amplified spontaneous emission (ASE) from the laser that couples to other excited states. % to allow perfect overlap and to eliminate additional beams and therefore scattering sources.
After the total tweezer power is set to the desired value, we smoothly ramp down the power of one frequency in the tweezer while simultaneously ramping up the power of a different frequency so that the total tweezer power remains unchanged. Both frequencies are kept on for a specified duration before the process is reversed and the tweezer returns to a single frequency.
% In addition to reducing the number of laser beams during the Raman transition,


% found the spectral purity of the laser for Raman beams to be critical for achieving a higher transfer efficiency.
% The tweezer/Raman beams are generated by a fiber amplifier seeded with a $1037~\mathrm{nm}$ external cavity diode laser.
% by amplifying a $1037~\mathrm{nm}$ external cavity diode laser (ECDL) with a fiber amplifier,
% The diode laser produces the desired frequency on top of a broad band amplified spontaneous emission (ASE).  We use a Bragg grating with a line width (FWHM) of $50~\mathrm{GHz}$ to clean up the laser spectrum and found it reduces the scattering rate by at least a factor of $2$
% for a particular single photon detuning.

% in addition to the desired frequency. This increases the scattering rate due to coupling to other excited states.

\begin{figure}[t!]
  \includegraphics[width=0.48\textwidth]{fig3.pdf}
  \caption{
    (a) Direct measurement of molecule lifetime in $3.75~\mathrm{mW}$ of trap depth.
    Molecule survival is detected by dissociating back to atoms using a second Raman transition.
    The lifetime is consistent with the $0.199(9)~\mathrm{ms}$
    measured from the Raman transition data.
    (b) Two-body atom lifetime of $5(1)~\mathrm{ms}$
    in $3.75~\mathrm{mW}$ of trap depth caused by off-resonance photoassociation.
    This is used to improve the fitting of the Raman transfer data.
    Inset: Atomic scattering rate scales as $P_{tweezer}^{2.58}$ on a log-log scale,
    this is consistent with a two-photon scattering process.
    We have not measured a clear dependency of the loss rate on the tweezer detuning.
    \label{f-lifetime}}
\end{figure}

%% Experiment condition/resonance

We choose the tweezer frequency to be far detuned ($145 ~\mathrm{GHz}$) from the $v' = 0$ line, and guided by coupled channel calculations,
we locate the Raman resonance for the atom to molecule transition at
$770.57150(9) ~\mathrm{MHz}$ (Fig.~\ref{f-raman}a)
with $3.75~\mathrm{mW}$ tweezer power.
% (We can maybe add information about the prediction here?)
%% Prediction
% This excited state used in the Raman transition was measured in our previous experiment using photoassociation to be at $288560 GHz$ from our atomic state. The ground molecular state has not been observed previously in experimentally. Based on our measurement of FB resonance, interaction shift and the binding energy of the 4422 bound state. Theory prediction was at $770.1 MHz$.
% The background level of $31~\mathrm{\%}$ corresponds to the probability of preparing the two atoms in the relative motional ground state.
% When the atoms are transferred into the molecule state by the Raman transition, there is a decrease in the two body survival since the resulting molecule is dark to our imaging sequence. % directly detected by our imaging step.
The molecular state is dark to the imaging step,
so successful transfer of the atoms to the molecular state will be detected as loss.
We observed a Fourier limited linewidth which is evidence of a coherent transfer.
In order to verify the coherence of the transfer directly,
we fix the two-photon detuning on resonance and scan the pulse time.
Fig.~\ref{f-raman}b shows the observed coherent Rabi oscillation between the atomic and molecular states.
Fitting the data with a decaying Rabi oscillation suggests that
$69~\mathrm{\%}$ of initial ground state atoms are transferred into the molecular state after a $\pi$ pulse.
This transfer efficiency is mainly limited by the molecular lifetime
which can be measured directly by preparing the molecule with a $\pi $ pulse
and then using a second $\pi$ pulse to dissociate the molecule back to atoms
after a variable wait time (Fig.~\ref{f-lifetime}a).
The result shows a molecular lifetime of $0.199(9)~\mathrm{ms}$
consistent with the decay of Rabi oscillation.
We obtain the Raman Rabi frequency by fitting our measurements to a model
that includes a Raman Rabi frequency and
a finite lifetime for the molecular state (Fig.~\ref{f-raman}a and b).
We account for the effect of atomic state loss by measuring the one and
two body lifetime of the atoms directly (Fig.~\ref{f-lifetime}b)
without turning on the second frequency.
The fit shows that we have a Raman Rabi frequency of $2\pi\times3.28(4) ~\mathrm{kHz}$.

The efficiency of the transfer is lower than expected from theory and arises from the ratio of the molecule scattering rate to the Rabi frequency being 10 times larger than predicted.
Based on the discussion above, if this discrepancy arises from the $v'=0$ excited state,
it can be either due to a high ratio of $\Omega_m / \Omega_a$ or a large $\Gamma_e$.
Additionally, coupling to other excited states can also add an offset to
both the Raman Rabi frequency and the scattering rate
which can affect the scattering rate to Raman Rabi frequency ratio.

In order to verify whether any one of these known sources are the origin of the discrepancy,
we measured the properties of the Raman resonance as a function of the tweezer power and single photon detuning.
These dependencies allow us to experimentally determine the matrix elements,
$ \Omega_a $, $\Omega_m $ and how much of the scattering, Stark shift,
or Raman Rabi frequency comes from the $ v' = 0$ intermediate state.

First we look at the change in resonance frequency.
As a function of the tweezer power,
we observe the expected linear dependency on the resonance frequency
caused by the differential light shift between the atomic and molecular state.
When we vary the tweezer frequency around the $v'=0$ intermediate state,
we can further observe a $1/\Delta$ component
and a constant background in the experimentally explored region (Fig.~\ref{f-det}a).
The background is caused by coupling to other excited states that are further away in energy.
The $1/\Delta$ component, however, is due to the coupling between the molecular state
and the $v'=0$ intermediate state.
From this measurement, we can extract a $\Omega_m$ of
$2\pi\times72.32(4) ~\mathrm{MHz}/\sqrt{\mathrm{mW}}$ or
$2\pi\times140.06(8) ~\mathrm{MHz}$ for the $3.75~\mathrm{mW}$ tweezer power used above.
This number is close to the value of
$2\pi\times40.7 ~\mathrm{MHz}/\sqrt{\mathrm{mW}}$ calculated from theory. \todo{ref/sm theory}

\begin{figure*}
  \includegraphics[width=\textwidth]{fig4.pdf}
  \caption{Raman transition parameters as a function of tweezer and Raman power and detuning.
    The detuning is calculated from the closest $v'=0$ PA frequency at $288703.6~\mathrm{GHz}$.
    (a) The light shift of the Raman resonance scales as $P$
    and follows $1/\Delta$ with an offset, where $ P$ is the power in the tweezer.
    The fit also includes a small term that is proportional to $P^2$
    which is caused by the effective magnetic field generated by the tweezer which is
    perpendicular to the real magnetic field.
    (b) Raman Rabi frequency ($\Omega_R$) scales as $P^{1.29}$
    and follows $1/\Delta$ with an offset.
    (c) Tweezer power dependency of light shift (up) and Raman Rabi frequency (down)
    on a log-log plot showing the power law scaling.
    \label{f-det}}
\end{figure*}

In order to calculate the $\Omega_m/\Omega_a$ ratio,
we now need to extract $ \Omega_a $.
We do this by measuring the dependencies of the Raman Rabi frequency,
which depends on both $\Omega_m$ and $\Omega_a$.
The Raman Rabi frequency shows a non-linear dependency on the tweezer power
due to the change in the atomic wavefunction caused by
tighter confinement at higher power (Fig.~\ref{f-det}b).
As discussed before, for weakly interacting particles,
$\Omega_a$ scales as $P^{0.375}$.
However, due to the strong interaction between the two atoms, this approximation breaks down.
Instead, coupled-channel calculations show that the scaling
is very well approximated by $P^{0.29}$ within the range of confinement in our experiment.
Combined with the standard intensity factor, the Raman Rabi frequency should scale as $P^{1.29}$,
which fits well to our experimental result (Fig.~\ref{f-det}c).
Similar to the light shift, the Raman Rabi frequency's detuning dependence is determined by a constant background component and
a $v'=0$ component in the Raman Rabi frequency that scales as $1/\Delta$.
The $v'=0$ component of the Raman Rabi frequency is
$2\pi\times1.02(2)~\mathrm{kHz\cdot mW^{-1.29}}$,
or $2\pi \times 5.6(1)~\mathrm{kHz}$ at $3.75~\mathrm{mW}$ tweezer power.
Together with the $\Omega_m$ measured above, the Rabi frequency, $ \Omega_a $, is
$2\pi \times 12.1(3)~\mathrm{MHz}$.
This gives a matrix element ratio of $11.6(3)$,
which is in fact better than the theory prediction of $23.8$ (Table \ref{tab:rabi_freqs}).
Therefore, this should not cause the ratio of the Raman Rabi frequency to scattering rate
from the $v'=0$ state to be higher than expected.
Furthermore, we independently measure the natural linewidth of the $v'=0$ excited state to be no larger than $20~\mathrm{MHz}$ using photoassociation (PA) spectroscopy.
This suggests that the excited state linewidth should not cause
a stronger than expected scattering from $v'=0$ state either.

\begin{table}[h]
    \centering
    \begin{tabular}{c|c|c}
         & Experiment & Theory \\ \hline
         $ \Omega_m $ & $2\pi \times 140.06(8) ~\mathrm{MHz} $ & $2\pi \times 78.9 ~\mathrm{MHz}$ \\
       $\Omega_a$ & $2\pi \times 12.1(3) ~\mathrm{MHz}$ & $2\pi \times 3.32 ~\mathrm{MHz}$ \\
       $\Omega_m/\Omega_a$ & $11.6(3)$ & $ 23.8$
    \end{tabular}
    \caption{Comparison between theory and experiment of the Rabi frequencies $\Omega_m$ and $\Omega_a$ at $3.75$ mW tweezer power. The experimentally measured $\Omega_m/\Omega_a$ ratio is smaller than the theory prediction.}
    \label{tab:rabi_freqs}
\end{table}

With the $ v' = 0 $ state ruled out as the source of discrepancy between experiment and theory,
we now consider the background effects from other states with larger single photon detuning.
In the Raman Rabi frequency fit, the fitted background is of an opposite sign from the Raman Rabi frequency for single photon detunings red of the $v' = 0 $ transition. Thus, this background reduces the Raman Rabi frequency by about $30~\mathrm{\%}$ at the current detuning.
However, this difference is not enough to explain the greater than a factor of $10$ discrepancy
present in the experiment.
Due to the change in sign of the Raman Rabi frequency as a function of detuning when crossing a resonance, the same background will increase the Raman Rabi frequency for detunings blue of the $v' = 0$ transition, which increases the Raman Rabi frequency. Unfortunately, we have observed additional nearby excited states belonging to a different excited state potential at higher frequencies
which prevent the blue side of the transition to be usable for the Raman transition.

\todo{change scattering to decoherence? since the fluctuation of light shift
  does not lead to scattering but only decherence.}
These results suggest that the decoherence or loss we observed during the Raman transition
comes from either a higher than expected background scattering rate
or a different intrinsic or technical source that we have not accounted for.
We have observed a decrease in the coherence time by a factor of 2 without the ASE filter,
suggesting the spectral purity of the laser is a significant source of scattering.
Other sources that can contribute to the decoherence includes
the stability of the tweezer power and the magnetic field.
Based on the Raman Rabi frequency to light shift ratio (Fig.~\ref{f-det}c),
the requirement on the tweezer power stability is $0.8~\mathrm{\%}$ at $3.75~\mathrm{mW}$, which we stabilize to $0.1~\mathrm{\%}$ so this should not be a major source
of decoherence.
Similarly, we measured a Zeeman shift of $42.2(2)~\mathrm{kHz/G}$
which does not cause significant decoherence from the measured magnetic field
fluctuation of $\sim1.5~\mathrm{mG}$.

The scattering rate of the molecule also depends on the tweezer power and detuning, and becomes smaller when the power is lowered.
At $0.75~\mathrm{mW}$ tweezer power, we have observed
a molecule lifetime as long as $1~\mathrm{ms}$.
However, since the technical noise that can lead to decoherence
is not fully characterized in our experiment,
we are unable to further identify the sources of the measured scattering rate
based on our measured detuning and power dependencies.

Lastly, to confirm that the excess scattering does not come from the atomic state,
we measure the two-body scattering rate
without turning on the second frequency (Fig.~\ref{f-lifetime}b inset).
The scattering rate scales as $P_{\mathrm{tweezer}}^{2.58}$ which is inconsistent
with a single photon scattering process.
We have not been able to observe a dependency on the detuning in order to verify
if the scattering process is related to the $v'=0$ state,
but the power scaling strongly suggests the existence of an unknown two-photon process.
Nevertheless, the absolute scattering rate from the atomic state
is much lower than the total scattering rate
and is not the limiting factor in this experiment.

In conclusion, we have formed a weakly bound NaCs molecule in an optical tweezer
via an optical Raman transfer.
A theoretical investigation including all excited states of $\mathrm{c^3\Sigma^+}(\Omega = 1)$,
the excited atomic continuum, and coupled channel ground state wavefunctions indicated
better transfer efficiency using a deeply bound intermediate state
and the $\ket{\uparrow_{\Na} \downarrow_{\Cs} }$ spin state as the initial and final states.
Using these theoretical insights, we located the weakly bound state
and coherently associated the atoms into a weakly bound molecule.
Our transfer efficiency is limited by an unknown scattering source
resulting in measured scattering rates over $10$ times larger than theoretical predictions.
Despite this limitation, the transfer efficiency may be further improved
by increasing the ratio of the up-leg to down-leg Rabi frequency
$\Omega_a/\Omega_m$ by exploring the possibility of
driving to more deeply bound states.
There may also be a better choice of single photon detuning to increase the Raman Rabi frequency,
since our location results in about $30~\mathrm{\%}$ cancellation of the Raman Rabi frequency
due to an offset of unknown origin.

Our technique can be applied to form a more diverse set of molecular species,
since it does not rely on a magnetic Feshbach resonance, bound states at the MHz-level,
or a narrow excited state. The formation of a weakly-bound molecule is a key step
in forming rovibrational ground state molecules.
Combined with real time rearrangement \cite{Barredo2016, Endres2016},
defect free arrays of highly controlled molecules comprise a promising
and flexible platform for quantum simulation and quantum computing applications.

\todo{sm: STIRAP vs Raman}

\begin{acknowledgments}
  We would like to thank Bo Gao,  Paul Julienne, and Rosario Gonzalez-Ferez for discussion. This work is supported by the NSF~(PHY-1806595), the AFOSR~(FA9550-19-1-0089), ARO DURIP (W911NF1810194) and the Arnold and Mabel Beckman foundation. J.~T.~Z. is supported by a National Defense Science and Engineering Graduate Fellowship. W.~C. is supported by a Max Planck-Harvard Research Center for Quantum Optics fellowship. K.~W. is supported by an NSF GRFP fellowship. J.~M.~H. is supported by the U.K. Engineering and Physical Sciences Research Council (EPSRC) Grants No.\ EP/N007085/1, EP/P008275/1 and EP/P01058X/1.
\end{acknowledgments}

\bibliography{master_ref}
\bibliographystyle{apsrev4-2}
\end{document}
