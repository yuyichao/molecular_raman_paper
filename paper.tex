\documentclass[aps,prl,twocolumn,groupedaddress]{revtex4-1}
% \documentclass[aps,twocolumn,secnumarabic,balancelastpage,amsmath,amssymb,nofootinbib]{revtex4-1}
\usepackage{amsmath}
\usepackage{amssymb}
\usepackage{amsfonts}
\usepackage{color}
\usepackage{graphics}
\usepackage[pdftex]{graphicx}
\usepackage[utf8x]{inputenc}
\usepackage[colorlinks=true]{hyperref}

\newcommand{\ud}{\mathrm{d}}
\newcommand{\ue}{\mathrm{e}}
\newcommand{\ui}{\mathrm{i}}
\newcommand{\res}{\mathrm{Res}}
\newcommand{\Tr}{\mathrm{Tr}}
\newcommand{\dsum}{\displaystyle\sum}
\newcommand{\dprod}{\displaystyle\prod}
\newcommand{\dlim}{\displaystyle\lim}
\newcommand{\dint}{\displaystyle\int}
\newcommand{\fsno}[1]{{\!\not\!{#1}}}
\newcommand{\texp}[2]{\ensuremath{{#1}\times10^{#2}}}
\newcommand{\dexp}[2]{\ensuremath{{#1}\cdot10^{#2}}}
\newcommand{\eval}[2]{{\left.{#1}\right|_{#2}}}
\newcommand{\paren}[1]{{\left({#1}\right)}}
\newcommand{\lparen}[1]{{\left({#1}\right.}}
\newcommand{\rparen}[1]{{\left.{#1}\right)}}
\newcommand{\abs}[1]{{\left|{#1}\right|}}
\newcommand{\sqr}[1]{{\left[{#1}\right]}}
\newcommand{\crly}[1]{{\left\{{#1}\right\}}}
\newcommand{\angl}[1]{{\left\langle{#1}\right\rangle}}
\newcommand{\tpdiff}[4][{}]{{\paren{\frac{\partial^{#1} {#2}}{\partial {#3}{}^{#1}}}_{#4}}}
\newcommand{\tpsdiff}[4][{}]{{\paren{\frac{\partial^{#1}}{\partial {#3}{}^{#1}}{#2}}_{#4}}}
\newcommand{\pdiff}[3][{}]{{\frac{\partial^{#1} {#2}}{\partial {#3}{}^{#1}}}}
\newcommand{\diff}[3][{}]{{\frac{\ud^{#1} {#2}}{\ud {#3}{}^{#1}}}}
\newcommand{\psdiff}[3][{}]{{\frac{\partial^{#1}}{\partial {#3}{}^{#1}} {#2}}}
\newcommand{\sdiff}[3][{}]{{\frac{\ud^{#1}}{\ud {#3}{}^{#1}} {#2}}}
\newcommand{\tpddiff}[4][{}]{{\left(\dfrac{\partial^{#1} {#2}}{\partial {#3}{}^{#1}}\right)_{#4}}}
\newcommand{\tpsddiff}[4][{}]{{\paren{\dfrac{\partial^{#1}}{\partial {#3}{}^{#1}}{#2}}_{#4}}}
\newcommand{\pddiff}[3][{}]{{\dfrac{\partial^{#1} {#2}}{\partial {#3}{}^{#1}}}}
\newcommand{\ddiff}[3][{}]{{\dfrac{\ud^{#1} {#2}}{\ud {#3}{}^{#1}}}}
\newcommand{\psddiff}[3][{}]{{\frac{\partial^{#1}}{\partial{}^{#1} {#3}} {#2}}}
\newcommand{\sddiff}[3][{}]{{\frac{\ud^{#1}}{\ud {#3}{}^{#1}} {#2}}}
\newcommand{\eff}{ef\! f}
\newcommand{\fxnote}[1]{{\textbf{[#1]}}}

\newcommand{\todo}[1]{}

\ifpdf
% Ensure reproducible output
\pdfinfoomitdate=1
\pdfsuppressptexinfo=-1
\pdftrailerid{}
\hypersetup{
  pdfcreator={},
  pdfproducer={}
}
\fi

\begin{document}
\title{Coherent optical association of single molecules}
\author{Yichao Yu}
\email{yichaoyu@g.harvard.edu}
\author{Kenneth Wang}
\author{Jessie T. Zhang}
\author{Lewis Picard}
\author{William Cairncross}
\author{Kang-Kuen Ni}
\email{ni@chemistry.harvard.edu}
\affiliation{Department of Chemistry and Chemical Biology, Harvard University, Cambridge, Massachusetts, 02138, USA}
\affiliation{Department of Physics, Harvard University, Cambridge, Massachusetts, 02138, USA}
\affiliation{Harvard-MIT Center for Ultracold Atoms, Cambridge, Massachusetts, 02138, USA}

\date{\today}

\begin{abstract}
  We report coherent association of a single NaCs molecule in an optical tweezer
  through an optical Raman transition.
  By selecting a deeply bound intermediate state,
  we suppress the scattering loss during the transfer process.
  Starting from atoms in their relative motional ground state,
  we achieve optical transfer efficiency of $50 \%$ \todo{number}.
  The molecule we create have a zero-field binding energy of $770 \mathrm{MHz}$ \todo{number}
  and lifetime up to $1 \mathrm{ms}$ \todo{number}.
  We demonstrate that coherent creation of ground state single molecule is possible,
  even without Feshbach resonance or narrow optical transition.
\end{abstract}

% Coherent
% Intermediate state
% Motional state control
% Initial state preparation

\maketitle

% Introduction

Trapped neutral molecules, assembled in an array of optical tweezers,
are a promising platform to study quantum infoqrmation and quantum simulations.

\begin{figure*}
  \includegraphics[height=4.5cm]{fig1.pdf}
  \caption{Optical creation of single molecule from single atoms in tweezer.
    (A) Schematics of the Raman transition. \todo{more about the states involved}
    (B) Geometry and polarization of trap and Raman beam relative to the bias magnetic field.
    \todo{Field strength, power, polarization description (pi)}
    (C) Molecule formation pulse sequence. The tweezer initially consists of only up leg power.
    This power is smoothly ramped down and the down leg power ramped up over $10\mu s$ while
    maintaining the total power of the tweezer.
    \todo{to minimize heating?}
    \label{f-setup}
  }
\end{figure*}

\begin{figure*}
  \includegraphics[height=4.5cm]{fig2.pdf}
  \caption{
    (A) Raman resonance from atomic state to molecular state, showing Fourier limited linewidth.
    \todo{states, time}
    (B) Rabi oscillation on resonance
    \label{f-raman}}
\end{figure*}

\begin{figure*}
  \includegraphics[height=4.5cm]{fig3.pdf}
  \caption{
    (A) Molecule lifetime in 15 mW of trap depth \todo{lifetime number}
    (B) Two-body atom lifetime in 15 mW of trap depth \todo{lifetime number,
      subtraction of single body, photoassociation rate}
    \label{f-lifetime}}
\end{figure*}

In this Letter, we demostrate that a single ground electronic state molecule can be formed
from two single optical tweezers by driving an optical Raman transition

\todo{Previous result}
\todo{talk about size mismatch, two step transfer/weakly bound molecule in intro}

%% Preparation
Our experiment begins by loading a single ${}^{23}\mathrm{Na}$ atom and a single ${}^{133}\mathrm{Cs}$ atom into an optical tweezer from a dual-species MOT\todo{cite na loading paper} into separate optical tweezers. The atoms are imaged to distinguish between loading of two atoms, one atom (Na or Cs), or no atom during post selection. We then perform simutanious Raman sideband cooling (RSC) to cool both atoms into a single 3-dimentional motional ground state in the tweezers. After RSC, the Na tweezer is moved by sweeping the frequency on an acustical optical beam deflector (AOBD) to overlap with the Cs tweezer before smoothly ramping off so that the Na and Cs atoms are merged into the same tweezer \todo{cite}. The spin states for the Na and Cs atoms after RSC and during the merge process are $|F=2,m_F=2\rangle$ and $|F=4,m_F=4\rangle$ respectively. This states combination has a low scattering length of $... a_0$ which allows the two atoms to be merged into the same tweezer with minimum pertubation on each other and remains in their motional ground state after the merge.

After preparing the Na and Cs atoms in the same tweezer in a single quantum state, we perform interaction shift spectroscopy using a Cs Raman transition to drive the Cs into the $|F=3,m_F=3\rangle$ state. The new spin state combination has a larger scattering length of $... a_0$ which generates a interaction shift of $... kHz$ in the tweezer. This interaction shift is larger than the differential axial trapping frequency between Na and Cs atoms, which decouples the relative and center of mass motional state and improves the robustness of our preparation of relative motional ground state. The stronger interaction in this spin state also enhances the atomic wavefunction at short range and increases its overlap with the intermediate molecular state used for our Raman transfer.

%% Transfer scheme

The optical transfer scheme we used is shown in figure \todo{}.
We use an optical Raman transition to drive the system from the atomic initial state to a
ground electronic molecular state.
In order to reduce the size mismatch between the atomic and molecular states,
we selected the first bound state for the 3322 spin state \todo{asymtopt to the 3322 threshold?} as our final state. \todo{move selection of intermediate state to intro?}.
For similar reason, the natural choice for the intermediate excited molecular state is one
with highly excited motional level.
However, from our calculation (and experiment?), the smaller level space for high
vibrational state and the smaller detuning from the atomic threshold increases the
the scattering rate of the molecular state which causes a reduced Raman Rabi frequency
to decoherence rate ratio and a lower transfer efficiency.
Therefore, in our experiment, we selected the v=0 state as the intermediate state for our
Raman transition.

The pulse sequence for the experiment can be seen in figure (). Instead of adding another beam to drive the Raman transition on the atoms in the tweezer, we use the tweezer itself to achieve this goal. After the total tweezer power is set to the desired value, we smoothly ramp down the power of one frequency in the tweezer while simutaniously ramping up the power of a different frequency so that the total tweezer power remains unchanged. Both frequencies are kept on for a variable length of time before the process is reversed and we return to having a single frequency in the tweezer. The dual use of the tweezer beam ensures that there is not any undesired laser frequency that can interfere with the Raman transition. Using the tweezer beam for the Raman transition also allow us to maximize the intensity of the Raman beams and minimize the transfer time.

Figure () shows the geometry of the experiment setup. The B field which defines the quantization axis in the experiment is perpendicular to the tweezer axis. The tweezer beam (both frequencies for the Raman transition) has a $\pi$ polarization relative to the quantization axis, which allows us to selectively drive the Raman transition to a final state with the same total $m_F$ quantum number as our initial state.

%% Prediction

This excited state used in the Raman transition was measured in our previous experiment using PA to be at $288... GHz$ from our atomic state. The ground molecular state has not been observed previously in experimentally. Based on previous FB and interaction shift and 4422 bound state data. Theory prediction was at $770... MHz$. \todo{more, mention/cite Jeremy}

%% Experiment condition/resonance

\todo{Resonance observed at}
\todo{FWHM of line}

\todo{Rabi flopping}

\todo{Scattering}

\bibliography{paper}
\end{document}
